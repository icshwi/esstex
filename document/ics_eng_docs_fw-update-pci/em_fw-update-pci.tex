\documentclass[11pt
  , a4paper
  , article
  , oneside
  %  , twoside
  , showtrims
 % , draft
]{memoir}

\usepackage{essdocs}
\usepackage[numbers]{natbib}
\usepackage[autostyle]{csquotes}

\setsecnumdepth{subsection}


\begin{document}
%\frontmatter
% ESS Document Description
%
\essdoctype{Engineering Manual}

% ESS Document Number
%
\essdocnum{ESS-0870816}

% Date
%
\date{\today}

% ESS Document Revision Number
%
\essdocrev{1}

% ESS Document State
%
\essdocstate{Released}

% ESS Document Classification
%
\essdocclass{Internal}

% Document Title
%
\title{ICS Engineering Manual}
\subtitle{for Firmware update of PCI-based MRF products}
% Document Author(s), if more than one author,
% use \newline instead of \\ or \linebreak in order to seperate them
\author{Javier Cereijo Garcia \newline Jeong Han Lee}
\authorrole{Timing System Engineer \newline Senior Controls Engineer}

% Document Reviewer(s) if more than one reviewer,
% use \newline instead of \\ or \linebreak in order to seperate them
%\reviewer{-}
%\reviewerrole{-}

% Document Owner(s) if more than one owner,
% use \newline instead of \\ or \linebreak in order to seperate them
\owner{Javier Cereijo Garcia}
\ownerrole{Timing System Engineer}

% Document Approver(s) if more than one approver,
% use \newline instead of \\ or \linebreak in order to seperate them
%\approver{-}
%\approverrole{-}


\showtrimson

\esstitle
\newpage
\tableofcontents
\newpage

%\mainmatter


% Actual Document Start at below
\chapter{Overview}
At European Spallation Source (ESS), Integrated Control System (ICS) uses the MicroResearch Finland (MRF) Timing System{\footnote{\url{http://www.mrf.fi/}}} as its timing system of the ESS site. The consistent and up-to-date engineering manual is essential for the ESS timing system.\\

\section{Scope}
\begin{itemize}
\item This document explains how to perform a Firmware (FW) update of PCI-based MRF products used at ESS.
\item This document identifies two possible ways of performing such update: using the \texttt{mrfioc2} EPICS module, or using the vendor's own driver.
\item This document attempts to maintain consistency with existing ESS timing system hardware as far as possible.
\end{itemize}
%\textbf{Note that this is a very early draft document and should be updated as development progresses.}

\section{Target audience}
This document is targeted to ICS engineers and technical stakeholders of the ESS timing system. It is assumed that the target audience has a technical background in the MRF Timing System, the Experimental Physics and Industrial Control System (EPICS) development, and a Linux environment.\\

\chapter{System description}
MRF Technical Reference \citep[see][p5]{MRFEVENTSYSTEMDC} describes the timing system as:
\blockquote{\textit{The MRF Timing System provides a complete timing distribution system including timing signal generation with only a few components.\\The system is capable of generating and synchronous frequencies, trigger signals and sequences of events, etc. synchronous to an externally provided master clock reference and mains voltage phase signal. Support for timestamps makes the system a global timebase and allows attaching timestamps to collected data and performed actions.}}

ICS uses and will use the following different types of MRF products:
\begin{itemize}
\item MTCA-EVR-300
\item PCIe-EVR-300DC
\item MTCA-EVM-300
\end{itemize}

The scope of this document is to explain how to update the FW of all of them, but other products may also be updated using the same procedure (not tested).\\


%\clearpage

\chapter{System environment}
Before describing the engineering procedure for the FW update, it is mandatory to have a proper system environment that consists of specific hardware and software. The information shown in this chapter is used in the ICS lab at ESS.\\

\section{Hardware}
Table~\ref{table:hwlist} shows the hardware list and its environment. It is possible to use different form factors than what is shown in the examples; for more information check the specific engineering manual.\\

\begin{table}[!hb]
  \centering
  \begin{tabular}{l|l}
    \toprule
    Hardware                        & Info               \\\midrule
    MRF mTCA-EVM-300                &                    \\\midrule
    MRF PCIe-EVR-300DC              &                    \\\midrule
    MRF mTCA-EVR-300                &                    \\\midrule
    $\mu$TCA crate                  & Incl. PM, MCH      \\\midrule
    Concurrent Technologies AMC CPU &                    \\\midrule
    Adlink Industrial PC            &                    \\\midrule
    Optical cables                  & LC, optical 850 nm \\\bottomrule
  \end{tabular}
  \caption[]{Hardware List and its Environment.}
  \label{table:hwlist}
\end{table}

% \clearpage
\section{Software}
Table~\ref{table:swlist} shows the software list and its environment.\\

\begin{table}[!htb]
  \centering
  \begin{tabular}{l|l}
    \toprule
    Item               & Version Info.                                                      \\\midrule
    CentOS Linux       & \texttt{7.7.1908}                                                  \\\midrule
    Kernel             & \texttt{3.10.0-1062.9.1.el7.x86\_64}                               \\\midrule
    mrf kernel module  & Version: \texttt{1} / srcversion \texttt{E3290AD048B5B57D2EAA55E}  \\\midrule
    EPICS base         & \texttt{7.0.3.1}                                                   \\\midrule
    e3-req             & \texttt{3.1.2}                                                     \\\midrule
    mrfioc2            & E3 module ver. \texttt{2.2.0-rc7}                                  \\\midrule
    devLib2            & E3 module ver. \texttt{2.9.0}                                      \\\bottomrule
  \end{tabular}
  \caption[]{Software and its version information.}
  \label{table:swlist}
\end{table}


%\clearpage

\chapter{Engineering procedure}
This chapter provides the instructions on how to perform the FW update.\\

\section{FW update using \texttt{mrfioc2}}
\textbf{This is the preferred method}, which we have tested with the MTCA-EVR-300(U), PCIE-EVR-300DC and MTCA-EVM-300. Since this method uses the \texttt{mrfioc2} module, it is assumed that the ESS EPICS environment (E3) is installed and loaded in the current session{\footnote{To install and load E3 check \url{https://github.com/icshwi/e3} and \url{https://github.com/icshwi/e3training}.}}. It is also necessary to have the \texttt{mrf} kernel module loaded and to register the EVR with the Input/Out Controller (IOC). To do this you need to know the PCI parameters, which can be retrieved as is explained in the engineering manual corresponding to your specific form factor \cite{empcieeevr300dc, emmtcaevr300, emmtcaevm300}.\\

Then run the following IOC start-up script from the directory containing your bitfile using your PCI and \texttt{require} parameters with \texttt{iocsh.bash fwupdate.cmd}, which is in \texttt{e3-mrfioc/cmds}:
\begin{lstlisting}[
  style=termstyle,
  numbers=left,
  label={list:fwupdate.cmd},
  caption={Start-up script \texttt{fwupdate.cmd}. Line \ref{require} and Line \ref{pciid0} should be modified according to the installed require version and its corresponding MRF hardware. One can get them easily via \texttt{bash /tools/get\_pciaddr.bash pcieevr300} and \texttt{make dep} within the \texttt{e3-mrfioc2} directory. }
]
require mrfioc2,2.2.0-rc7 (*@\label{require}@*)
epicsEnvSet("DEV1", "EVR1")
mrmEvrSetupPCI("$(DEV1)",  "0a:00.0") (*@\label{pciid0}@*)
iocInit()
\end{lstlisting}

Before upgrading, it is suggested to backup the existing FW. All Xilinx bit files for a particular device typically have the same size, so you can use the size of the new FW to estimate the size of the old one.\\

In this example of a PCIe-EVR-300DC with the 207.0 FW, the exact size is 3011417 bytes, which is arbitrarily rounded up to 3MB. Create the backup bit file running from inside the IOC:
\begin{lstlisting}[style=termstyle]
> flashread("EVR1:FLASH", 0, 0x300000, "PCIe-EVR-300DC.207.0.backup.bit")
| 3080192
| 3014656
| 2949120
  ...
| 65536
| 0

Done
epics>
\end{lstlisting}

Now write the new FW file, named \texttt{PCIe-EVR-300DC-17080207.bit} in this example, that should be located in the current directory. One can check this via \texttt{> system("ls *.bit")}:
\begin{lstlisting}[style=termstyle]
>  flashwrite("EVR1:FLASH", 0, "PCIe-EVR-300DC-17080207.bit")
| 0/3011417
| 65536/3011417
| 131072/3011417
| 196608/3011417
| 262144/3011417
...
...
| 2949120/3011417
| 3011417/3011417

Done
>
\end{lstlisting}

If the update process is interrupted, \textbf{DO NOT POWER CYCLE!} and re-run the update process in order to be completed. After the write completes successfully, power cycle the MRF EVR card to load the new bit file. For more information you can check \citep[section Firmware Update, PCIe-EVR-300DC, mTCA-EVR-300]{EVRUSAGEGUIDE}.\\


\subsection{Troubleshooting}
If the \texttt{mrfioc2} module was built for a different kernel version that the one the system has, it is possible that the FW update fails while printing a successful message. One can check the kernel module version via \texttt{modinfo mrf}. In this case, if the update is interrupted and the board is power cycled, or if you have any other problem that causes an incorrect bit file being flashed, bring the board to ICS for manual flashing.\\


%\clearpage
\section{FW update using the vendor's own driver}
It's better to update the FW with the previous method. This other option is only possible if we can write and compile on the machine, for example with a physical development machine, and we can log in as the root user. For this method you will need to download the driver from \url{https://github.com/jpietari/mrf-linux-driver}.\\

First make sure that you power cycle the crate, and that no \texttt{mrf} kernel module is loaded. You can check this with \texttt{lsmod |grep mrf}. If the module is loaded unload it with \texttt{sudo rmmod mrf}. Then clone the driver to the machine with the board you want to update. In this example we work in the \texttt{/home/iocuser/FWupdate} directory. The bitfile we are flashing is \texttt{mTCA-EVR-300DC-18050207.bit} found in the same directory.\\

Build the driver:
\begin{lstlisting}[style=termstyle]
iocuser@cslab-ccpu-crate07: mrf-linux-driver (master)$ make
[...]
make[1]: Leaving directory `/usr/src/kernels/3.10.0-1062.9.1.el7.x86_64'
\end{lstlisting}

Log in as root user and install and load the modules:
\begin{lstlisting}[style=termstyle]
iocuser@cslab-ccpu-crate07: mrf-linux-driver (master)$ sudo su
[sudo] password for iocuser:
[root@cslab-ccpu-crate07 mrf-linux-driver]# make modules_install
make -C /lib/modules/3.10.0-1062.9.1.el7.x86_64/build M=/home/iocuser/FWupdate/mrf-linux-driver modules_install
make[1]: Entering directory `/usr/src/kernels/3.10.0-1062.9.1.el7.x86_64'
  INSTALL /home/iocuser/FWupdate/mrf-linux-driver/pci_mrfevg.ko
Can't read private key
  INSTALL /home/iocuser/FWupdate/mrf-linux-driver/pci_mrfevr.ko
Can't read private key
  DEPMOD  3.10.0-1062.9.1.el7.x86_64
make[1]: Leaving directory `/usr/src/kernels/3.10.0-1062.9.1.el7.x86_64'
[root@cslab-ccpu-crate07 mrf-linux-driver]# depmod -a
[root@cslab-ccpu-crate07 mrf-linux-driver]# sh module_load
Found 0 Event Generators.
Found 1 Event Receivers.
\end{lstlisting}

Flash the bitfile to the EVR:
\begin{lstlisting}[style=termstyle]
[root@cslab-ccpu-crate07 mrf-linux-driver]# dd if=mTCA-EVR-300DC-18050207.bit of=/dev/era1
5881+1 records in
5881+1 records out
3011417 bytes (3.0 MB) copied, 11.2043 s, 269 kB/s
[root@cslab-ccpu-crate07 mrf-linux-driver]#
\end{lstlisting}

This will take some time. When it's finished power cycle the board to load the new FW.\\

For more information you can check \citep{MRFKERNELDRIVER}.\\

\subsection{Troubleshooting}
If the update is interrupted and the board is power cycled, or you have any other problem that causes an incorrect bit file being flashed, bring the board to ICS for manual flashing.\\


\clearpage


%\backmatter

\chapter*{Glossary}\label{sec:glossary}
\addcontentsline{toc}{chapter}{\nameref{sec:glossary}}
\begin{table}[!htb]
%  \footnotesize
%  \centering
  \begin{tabular}{ll}
    \toprule
    \textbf{Term} & Definition                                          \\\midrule
    E3            & ESS EPICS environment                               \\
    EPICS         & Experimental Physics and Industrial Control System  \\
    ESS           & European Spallation Source                          \\
    EVM           & Event Master                                        \\
    EVR           & Event Receiver                                      \\
    FW            & Firmware                                            \\
    ICS           & Integrated Control System                           \\
    IOC           & Input/Output Controller                             \\
    MRF           & MicroResearch Finland                               \\
    \bottomrule
  \end{tabular}
%  \caption[]{Document revision history.}
  \label{table:glossary}
\end{table}

%\bibliographystyle{unsrt}
%\bibliographystyle{plainnat}
%\bibliographystyle{abbrvnat}
\bibliographystyle{unsrtnat}
%\bibliographystyle{chicago}
\bibliography{./em_fw-update-pci}

\chapter*{Document revision history}\label{sec:docrevhist}
\addcontentsline{toc}{chapter}{\nameref{sec:docrevhist}}
\begin{table}[!htb]
  \footnotesize
  \centering
  \begin{tabular}{llll}
    \toprule
    \textbf{Revision} & \textbf{Reason for and description of change} & \textbf{Author}       & \textbf{Date} \\\midrule
    1                 & First release                                 & Javier Cereijo Garcia & \today        \\
    \bottomrule
  \end{tabular}
%  \caption[]{Document revision history.}
  \label{table:docrevhist}
\end{table}


\end{document}
