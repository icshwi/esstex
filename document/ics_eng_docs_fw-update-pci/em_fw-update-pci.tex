\documentclass[11pt
  , a4paper
  , article
  , oneside
  %  , twoside
  , showtrims
 % , draft
]{memoir}

\usepackage{essdocs}
\usepackage[numbers]{natbib}
\usepackage[autostyle]{csquotes}

%% \usepackage{epstopdf}
%% \usepackage[pdf]{pstricks}
%% \usepackage{graphicx}

\setsecnumdepth{subsection}

\begin{document}
%\frontmatter
%% ESS Document Description
%%
\essdocdesc{Engineering Manual}

%% ESS Document Number
%%
\essdocnum{ESS-XXXXXXXX}

%% Date
%%
\date{\today}

%% ESS Document Revision Number
%%
\essdocrev{0.2}

%% ESS Document State
%%
\essdocstate{Early Draft}

%% ESS Document Classification
%%
\essdocclass{ESS Use Only}

%% Document Title
%%
\title{ICS Engineering Manual}
\subtitle{for FirmWare update of PCI-based MRF products}
%% Document Author(s), if more than one author,
%% use \newline instead of \\ or \linebreak in order to seperate them
\author{Javier Cereijo Garcia \newline Jeong Han Lee }

%% Document Reviewer(s) if more than one reviewer,
%% use \newline instead of \\ or \linebreak in order to seperate them
%\reviewer{Timo Korhonen (Chief Engineer) \newline Timo Korhonen (Chief Engineer)}
\reviewer{TBD}
%% Document Owner(s) if more than one owner,
%% use \newline instead of \\ or \linebreak in order to seperate them
\owner{ICS}

%% Document Approver(s) if more than one approver,
%% use \newline instead of \\ or \linebreak in order to seperate them
\approver{ICS}

\showtrimson

\esstitle
\newpage
\tableofcontents
\newpage

%\mainmatter


%%% Actual Document Start at below
\chapter{Overview}
At European Spallation Source (ESS), Integrated Control System (ICS) uses the Micro Research Finland (MRF) Timing System{\footnote{\url{http://www.mrf.fi/}}} as its timing system of the ESS site. The consistent and up-to-date engineering manual is essential for the ESS Timing system.

\section{Scope}
\begin{itemize}
\item This document explains how to perform a FirmWare (FW) update of PCI-based MRF products used at ESS.
\item This document identifies two possible ways of performing such update: using the mrfioc2 EPICS module, or using the vendor's own driver.
\item This document attempts to maintain consistency with existing ESS Timing system hardware as far as possible.
\end{itemize}
\textbf{Note that this is a very early draft document and should be updated as development progresses.}

\section{Target Audience}
This document is targeted to ICS engineers and technical stakeholders of the ESS timing system. It is assumed that the target audience has a technical background in the MRF Timing System, the EPICS development, and a Linux environment.

\chapter{System Description}
MRF Technical Reference \citep[see][p5]{MRFEVENTSYSTEMDC} explained the timing system and wrote :
\blockquote{\textit{The MRF Timing System provides a complete timing distribution system including timing signal generation with only a few components.\\The system is capable of generating and synchronous frequencies, trigger signals and sequences of events, etc. synchronous to an externally provided master clock reference and mains voltage phase signal. Support for timestamps makes the system a global timebase and allows attaching timestamps to collected data and performed actions.}}

ICS uses and will use the following different types of MRF products:
\begin{itemize}
\item MTCA-EVR-300
\item PCIe-EVR-300DC
\end{itemize}

The scope of this document is to explain how to update the FW of both, but other products may also be updated using the same procedure (not tested).


\clearpage

\chapter{System Environment}
Before describing the engineering procedure for the FW update, it is mandatory to have proper system environment that consists of specific hardware and software lists.


\section{Hardware}
Table~\ref{table:hwlist} shows the hardware list and its environment. Here, \texttt{TAG} is used as the prefix of the ICS internal inventory system in order to track it down.
\begin{table}[!hb]
  \centering
  \begin{tabular}{l|l|l}
    \toprule
    Hardware                        & Info                                        & Serial Number \\\midrule
    MRF PCIe-EVR-300DC              & \texttt{ICS TAG-476}                        & M263060       \\\midrule
    Workstation                     &                                             &               \\\midrule
    MRF MTCA-EVR-300                & \texttt{ICS TAG-255}                        & L534006       \\\midrule
    NAT-MCH-PHYS                    & \texttt{ICS TAG-704}                        & 1135210809    \\\midrule
    Concurrent Technologies AMC CPU & \texttt{ICS TAG-734}, hostname: icslab-ts02 & M29597/010    \\\midrule
    Schroff MTCA 3U crate           & \texttt{ICS TAG-793}                        & 1401701204AA  \\\midrule
    NAT MTCA power supply unit      & \texttt{ICS TAG-743}                        & 121114 0269   \\\bottomrule
  \end{tabular}
  \caption[]{Hardware List and its Environment.}
  \label{table:hwlist}
\end{table}

% \clearpage
\section{Software}
Table~\ref{table:swlist} shows the Software list and its environment. It is mandatory to check the kernel version, and the mrf kernel module version. Since the mrfioc2 is dependent upon devlib2 E3 internally, an end-user is unnecessary to check its version explicitly.
\begin{table}[!htb]
  \centering
  \begin{tabular}{l|l}
    \toprule
    Item               & Version Info.                                                      \\\midrule
    CentOS Linux       & \texttt{CentOS 7.5.1804}                                           \\\midrule
    Kernel             & \texttt{3.10.0-862.3.2.el7.x86\_64}                                 \\\midrule
    mrf kernel module  & version : \texttt{1} / srcversion \texttt{A998B22F1425D7388F5F7A7} \\\midrule
    E3                 & \texttt{3.0.0}                                                     \\\midrule
    EPICS Base         & \texttt{3.15.5}                                                    \\\midrule
    mrfioc2            & E3 module ver. \texttt{2.2.0-rc2}                                  \\\midrule
    devLib2            & E3 module ver. \texttt{2.9.0}                                      \\\bottomrule
  \end{tabular}
  \caption[]{Software and its version information.}
  \label{table:swlist}
\end{table}


\clearpage
\chapter{Engineering Procedure}
This chapter provides the instructions on how to perform the FW update.


\section{FW update using mrfioc2}
This is the preferred method. Tested on MTCA, PCIe.

Since this method uses the mrfioc2 module, it is assumed that the E3 is installed and loaded in the current session{\footnote{To install and load E3 check \url{https://github.com/icshwi/e3} and \url{https://github.com/icshwi/e3training}.}}. It is also necessary to have the mrf kernel module loaded and to register the EVR with the IOC. To do so you need to know the PCI parameters, which can be retrieved as is explained in the engineering manual corresponding to your specific form factor \cite{empcieeevr300dc, emmtcaevr300}.

Then run the following IOC from the directory containing your bitfile using your PCI parameters with \texttt{iocsh.bash fwupdate.cmd}:
\begin{lstlisting}[
  style=termstyle,
  numbers=left,
  label={list:fwupdate.cmd},
  caption={Start-up script \texttt{fwupdate.cmd}. Line \ref{pciid0} should be modified according to the PCI parameters. }
]
require devlib2,2.9.0
require mrfioc2,2.2.0-rc2

epicsEnvSet("DEV1", "EVR1")

mrmEvrSetupPCI("$(DEV1)",  "01:00.0") (*@\label{pciid0}@*)

iocInit()

\end{lstlisting}

Before upgrading, it is suggested to backup the existing firmware. All Xilinx bit files for a particular device typically have the same size, so you can use the size of the new FW to estimate the size of the old one.

In this example of a PCIe-EVR-300DC with the 207.0 firmware, the exact size is 3011417 bytes, which is arbitrarily rounded up to 3MB. Create the backup bit file running from inside the IOC:
\begin{lstlisting}[style=termstyle]
epics> flashread("EVR1:FLASH", 0, 0x300000, "PCIe-EVR-300DC.207.0.backup.bit")
| 3080192
| 3014656
| 2949120
  ...
| 65536
| 0

Done
epics>
\end{lstlisting}

Now write the new FW file, named \texttt{PCIe-EVR-300DC.207.8.bit} in this example, that should be located in the current directory:
\begin{lstlisting}[style=termstyle]
epics> flashwrite("EVR1:FLASH", 0, "PCIe-EVR-300DC.207.8.bit")
| 0/3011417
| 65536/3011417
| 131072/3011417
...
| 2949120/3011417
| 3011417/3011417

Done
epics>
\end{lstlisting}

If the update process is interrupted, do not power cycle! Re-run the update process to completion.

After the write completes successfully, power cycle the card to load the new bit file.

For more information you can check \citep[section Firmware Update, PCIe-EVR-300DC, mTCA-EVR-300]{EVRUSAGEGUIDE}.


\subsection{Troubleshooting}
If the mrfioc2 module was built for a different kernel version that the one the system has, it is possible that the FW update fails while printing a successful message. In this case, if the update is interrupted and the board is power cycled, or if you have any other problem that causes an incorrect bit file being flashed, bring the board to ICS for manual flashing of the board.


\clearpage
\section{FW update using the vendor's own driver}
It's better to do it with the previous method. This one is only possible if we can write and compile on the machine, for example with a physical development machine, and we can log in as the root user.

This method was only tested on MTCA, not PCIe.

For this method you will need to ask ICS for the vendor's driver.

First make sure that you powercycle the crate, and that no mrf kernel module is loaded. You can check this with \texttt{lsmod |grep mrf}. If the module is loaded unload it with \texttt{sudo rmmod mrf}. Then copy the driver to the machine with the board you want to update. In this example we work in the \texttt{/home/iocuser/FWupdate} directory. The bitfile we are flashing is \texttt{mTCA-EVR-300DC-18050207.bit} found in the same directory.

Uncompress the driver and build it:
\begin{lstlisting}[style=termstyle]
[iocuser@icslab-ts02 FWupdate]$ tar -zxvf mrf_pci_driver.tar.gz
[...]
[iocuser@icslab-ts02 FWupdate]$ cd mrf_pci_driver/
[iocuser@icslab-ts02 mrf_pci_driver]$ make
[...]
\end{lstlisting}
Log in as root user and install and load the modules:
\begin{lstlisting}[style=termstyle]
[iocuser@icslab-ts02 mrf_pci_driver]$ sudo su
[sudo] password for iocuser:
[root@icslab-ts02 mrf_pci_driver]# make modules_install
make -C /lib/modules/3.10.0-229.7.2.el7.x86_64/build M=/home/iocuser/FWupdate/mrf_pci_driver modules_install
make[1]: Entering directory `/usr/src/kernels/3.10.0-229.7.2.el7.x86_64'
  INSTALL /home/iocuser/FWupdate/mrf_pci_driver/pci_mrfevg.ko
Can't read private key
  INSTALL /home/iocuser/FWupdate/mrf_pci_driver/pci_mrfevr.ko
Can't read private key
  DEPMOD  3.10.0-229.7.2.el7.x86_64
make[1]: Leaving directory `/usr/src/kernels/3.10.0-229.7.2.el7.x86_64'
[root@icslab-ts02 mrf_pci_driver]# depmod -a
[root@icslab-ts02 mrf_pci_driver]# sh module_load
Found 0 Event Generators.
Creating device nodes...
Found 1 Event Receivers.
Creating device nodes...
Creating nodes /dev/era[0-3] for major 247chgrp: invalid group: ‘mrf’
\end{lstlisting}
Flash the bitfile to the EVR:
\begin{lstlisting}[style=termstyle]
[root@icslab-ts02 mrf_pci_driver]# dd if=mTCA-EVR-300DC-18050207.bit of=/dev/era1
5881+1 records in
5881+1 records out
3011417 bytes (3.0 MB) copied, 11.2043 s, 269 kB/s
[root@icslab-ts02 mrf_pci_driver]#
\end{lstlisting}
This will take some time. When it's finished powercycle the board to load the new FW.

For more information you can check \citep{MRFKERNELDRIVER}.

\subsection{Troubleshooting}
If the update is interrupted and the board is power cycled, or you have any other problem that causes an incorrect bit file being flashed, bring the board to ICS and explained what has happened.



% TODO also show more sources: Jukka's doc, Michael's doc







\clearpage

\backmatter
%\bibliographystyle{unsrt}
%\bibliographystyle{plainnat}
%\bibliographystyle{abbrvnat}
\bibliographystyle{unsrtnat}
%\bibliographystyle{chicago}
%\bibliography{./ess_refs}
\bibliography{em_fw-update-pci}

\end{document}
